% !TeX spellcheck = de_DE
% Dieses Dokument muss mit PDFLatex gesetzt werden
% Vorteil: Grafiken koennen als jpg, png, ... verwendet werden
%          und die Links im Dokument sind auch gleich richtig
%
%Ermöglicht \\ bei der Titelseite (z.B. bei supervisor)
%Siehe https://github.com/latextemplates/uni-stuttgart-cs-cover/issues/4
\RequirePackage{kvoptions-patch}

%Englisch:
\let\ifdeutsch\iffalse
\let\ifenglisch\iftrue

%Deutsch:
%\let\ifdeutsch\iftrue
%\let\ifenglisch\iffalse

%
\ifdeutsch
	\PassOptionsToClass{numbers=noenddot}{scrbook}
\fi

%Warns about outdated packages and missing caption delcarations
%See https://www.ctan.org/pkg/nag
\RequirePackage[l2tabu, orthodox]{nag}

%Neue deutsche Trennmuster
%Siehe http://www.ctan.org/pkg/dehyph-exptl und http://projekte.dante.de/Trennmuster/WebHome
%Nur für pdflatex, nicht für lualatex
\RequirePackage{ifluatex}
\ifluatex
%do not load anything
\else
	\ifdeutsch
		\RequirePackage[ngerman=ngerman-x-latest]{hyphsubst}
	\fi
\fi

\documentclass[
               fontsize=12pt, %Default: 11pt, bei Linux Libertine zu klein zum Lesen
% BEGINN: Optionen für typearea
               paper=a4,
               twoside, % fuer die Betrachtung am Schirm ungeschickt
               BCOR=3mm, % Hack für BCOR (1.92 o.ä.), da bei BCOR2mm die Fuellpunkte beim Inhaltsverzeichnis falsch sind. Hack aber nicht mehr nötig: microtype für Verzeichnisse ausschalten hilft.
               DIV=13,   % je höher der DIV-Wert, desto mehr geht auf eine Seite. Gute werde sind zwischen DIV=12 und DIV=15
               headinclude=true,
               footinclude=false,
% ENDE: Optionen für typearea
%               titlepage,
               bibliography=totoc,
%               idxtotoc,   %Index ins Inhaltsverzeichnis
%                liststotoc, %List of X ins Inhaltsverzeichnis, mit liststotocnumbered werden die Abbildungsverzeichnisse nummeriert
               headsepline,
               cleardoublepage=empty,
               parskip=half,
%               draft    % um zu sehen, wo noch nachgebessert werden muss - wichtig, da Bindungskorrektur mit drin
               final   % ACHTUNG! - in pagestyle.tex noch Seitenstil anpassen
               ]{scrbook}


\input{preambel/packages_and_options}

%Der untere Rand darf "flattern"
\raggedbottom

%%%
% Wie tief wird das Inhaltsverzeichnis aufgeschlüsselt
% 0 --\chapter
% 1 --\section % fuer kuerzeres Inhaltsverzeichnis verwenden - oder minitoc benutzen
% 2 --\subsection
% 3 --\subsubsection
% 4 --\paragraph
\setcounter{tocdepth}{1}
%
%%%

\makeindex

%Angaben in die PDF-Infos uebernehmen
\makeatletter
\hypersetup{
            pdftitle={}, %Titel der Arbeit
            pdfauthor={}, %Author
            pdfkeywords={}, % CR-Klassifikation und ggf. weitere Stichworte
            pdfsubject={}
}
\makeatother


%%% acro
% alle Acronyme die verwendet werden kommen hier her.

\DeclareAcronym{ER}{short = ER , long = error rate}
\DeclareAcronym{FR}{short = FR , long = Fehlerrate}

%%%

\newenvironment{compact_itemize}{

 \begin{itemize}
  \setlength{\itemsep}{1pt}
  \setlength{\parskip}{0pt}
  \setlength{\parsep}{0pt}
}{
  \end{itemize}
 }

\newenvironment{keypoints}{%
\medskip\noindent
\begin{boxedminipage}{\linewidth}\vspace{6pt}
\begin{compact_itemize}}{%
\end{compact_itemize}\vspace{-6pt}
\end{boxedminipage}}


\begin{document}
	
%tex4ht-Konvertierung verschönern
\iftex4ht
% tell tex4ht to create picures also for formulas starting with '$'
% WARNING: a tex4ht run now takes forever!
\Configure{$}{\PicMath}{\EndPicMath}{} 
%$ % <- syntax highlighting fix for emacs
\Css{body {text-align:justify;}}

%conversion of .pdf to .png
\Configure{graphics*}  
         {pdf}  
         {\Needs{"convert \csname Gin@base\endcsname.pdf  
                               \csname Gin@base\endcsname.png"}%  
          \Picture[pict]{\csname Gin@base\endcsname.png}%  
         }  
\fi

%Tipp von http://goemonx.blogspot.de/2012/01/pdflatex-ligaturen-und-copynpaste.html
%siehe auch http://tex.stackexchange.com/questions/4397/make-ligatures-in-linux-libertine-copyable-and-searchable
%
%ONLY WORKS ON MiKTeX
%On other systems, download glyphtounicode.tex from http://pdftex.sarovar.org/misc/
%
\input glyphtounicode.tex
\pdfgentounicode=1

\VerbatimFootnotes %verbatim text in Fußnoten erlauben. Geht normalerweise nicht.

\input{macros/commands}
\pagenumbering{roman}
\Titelblatt

%Eigener Seitenstil fuer die Kurzfassung und das Inhaltsverzeichnis
\deftripstyle{preamble}{}{}{}{}{}{\pagemark}
%Doku zu deftripstyle: scrguide.pdf
\pagestyle{preamble}
\renewcommand*{\chapterpagestyle}{preamble}

%Kurzfassung / abstract
%auch im Stil vom Inhaltsverzeichnis
\ifdeutsch
\section*{Kurzfassung}
\else
\section*{Abstract}

This research is situated in the field of software engineering, specifically in authentication and authorization for web-based applications.
The focus is on integrating OAuth authentication into Gropius, an issue management tool aimed at unifying workflows across multiple platforms like GitHub and Jira.\\
 The current Gropius login service is stateless, requiring users to frequently re-authenticate and preventing seamless synchronization with external services.
This results in a limited user experience and hinders integration capabilities.\\
The objective of this research is to design and implement an OAuth-based authentication system for
Gropius that allows it to act both as an OAuth consumer (for external logins and data synchronization) and an OAuth provider
(enabling Gropius accounts to be used in other applications), while ensuring session-based access.\\
The research approach involves developing a new authentication service,
incorporating session management and token-based synchronization for cross-service data handling.
An appropriate database solution will be selected for secure token storage and session management, with consideration
given to scalability, performance, and compatibility with the overall system architecture.\\
 The implementation will enable session persistence, third-party authentication (e.g., GitHub and Google),
and token-based API synchronization for continuous data updates across platforms.
It will enhance Gropius’s interoperability and improve user experience by maintaining login states across sessions and devices.\\
Integrating OAuth into Gropius will significantly improve its authentication infrastructure,
providing a more scalable, secure, and user-friendly system. This implementation will allow Gropius to better support cross-platform workflows,
positioning it as a robust solution for unified issue management.

%\cleardoublepage


% BEGIN: Verzeichnisse

\iftex4ht
\else
\microtypesetup{protrusion=false}
\fi

%%%
% Literaturverzeichnis ins TOC mit aufnehmen, aber nur wenn nichts anderes mehr hilft!
% \addcontentsline{toc}{chapter}{Literaturverzeichnis}
%
% oder zB
%\addcontentsline{toc}{section}{Abkürzungsverzeichnis}
%
%%%

%Produce table of contents
%
%In case you have trouble with headings reaching into the page numbers, enable the following three lines.
%Hint by http://golatex.de/inhaltsverzeichnis-schreibt-ueber-rand-t3106.html
%
%\makeatletter
%\renewcommand{\@pnumwidth}{2em}
%\makeatother
%
%\tableofcontents

% Bei einem ungünstigen Seitenumbruch im Inhaltsverzeichnis, kann dieser mit
% \addtocontents{toc}{\protect\newpage}
% an der passenden Stelle im Fließtext erzwungen werden.

%\listoffigures
%\listoftables

%\ifdeutsch
%\printacronyms[name=Abkürzungsverzeichnis, heading=chapter*]
%\else
%\printacronyms[name=List of Acronyms, heading=chapter*]
%\fi

%Wird nur bei Verwendung von der lstlisting-Umgebung mit dem "caption"-Parameter benoetigt
%\lstlistoflistings 
%ansonsten:
%\ifdeutsch
%\listof{Listing}{Verzeichnis der Listings}
%\else
%\listof{Listing}{List of Listings}
%\fi

%mittels \newfloat wurde die Algorithmus-Gleitumgebung definiert.
%Mit folgendem Befehl werden alle floats dieses Typs ausgegeben
%\ifdeutsch
%\listof{Algorithmus}{Verzeichnis der Algorithmen}
%\else
%\listof{Algorithmus}{List of Algorithms}
%\fi
%\listofalgorithms %Ist nur für Algorithmen, die mittels \begin{algorithm} umschlossen werden, nötig

\iftex4ht
\else
%Optischen Randausgleich und Grauwertkorrektur wieder aktivieren
\microtypesetup{protrusion=true}
\fi

% END: Verzeichnisse

\mainmatter
\pagenumbering{arabic}

\renewcommand*{\chapterpagestyle}{scrplain}
\pagestyle{scrheadings}
\input{preambel/pagestyle}
%
%
% ** Hier wird der Text eingebunden **
%
% !TeX spellcheck = de_DE

\chapter{Introduction}


Authentication and authorization are critical components of modern software engineering, particularly in web-based applications that handle sensitive user data and integrate with external services.
OAuth \cite{SINGH2022103091}, an open standard for access delegation, has become a widely adopted solution for enabling secure authentication and seamless integration between platforms.
It allows applications to act as either consumers of third-party services (for features like social logins and data synchronization) or providers of authentication services for other systems.
This research focuses on integrating OAuth into Gropius \cite{gropius}, an issue management tool designed to unify workflows across platforms such as GitHub and Jira \cite{Atlassian, GitHub}.

The current Gropius login mechanism employs a stateless authentication service. As the current implementation does not use stateful sessions, the users have to frequently re-authenticate. \cite{GropiusGitHub}
This limitation impacts the user experience and creates barriers to integrating with external platforms, as there is no mechanism for persistent sessions.
Addressing these challenges is essential to enhance Gropius’s interoperability and provide a seamless experience for users who rely on multiple tools in their workflows.

To address these issues, this research aims to design and implement a new OAuth 2.1-based authentication system for Gropius.
This system will allow Gropius to function both as an OAuth consumer—enabling third-party logins and synchronization with external platforms like GitHub—and as an OAuth provider,
allowing other applications to authenticate users using their Gropius accounts.
The solution will incorporate session-based \cite{895139, 716693} access to overcome the limitations of the existing stateless service, ensuring that users remain logged in across sessions and devices.

The research methodology involves developing a robust authentication service that combines session management with token-based synchronization for cross-service data handling.
The implementation will consider scalability, security, and system compatibility, including the selection of a suitable database for managing session states and tokens.
This approach will enable stateful user sessions, third-party authentication (e.g., via GitHub and Jira),
and API-level data synchronization for real-time updates across integrated platforms.

The contribution of this research is the development of a comprehensive OAuth-based authentication system that enhances Gropius’s ability to integrate with external platforms and improves the overall user experience.
By enabling session persistence, supporting third-party logins, and facilitating cross-platform data synchronization,
this project positions Gropius as a more scalable, secure, and user-friendly tool for unified issue management.
This advancement will strengthen Gropius’s role as a central hub for managing workflows in diverse, interconnected environments.




\section*{Proposal Structure}
\begin{description}
	\item[Chapter~\ref{chap:foundation} -- \nameref{chap:foundation}:] Here, we provide the neccesary information about OAuth and the Gropius tool.
	\item[Chapter~\ref{chap:wp} -- \nameref{chap:wp}] Here, we present the objectives and work packages of our research project.
\end{description}

%\section*{Thesis Structure}
%Die Arbeit ist in folgender Weise gegliedert:
%\begin{description}
%\item[Kapitel~\ref{chap:ch2} -- \nameref{chap:ch2}:] Hier werden werden die Grundlagen dieser Arbeit beschrieben.
%\item[Kapitel~\ref{chap:conclusion} -- \nameref{chap:conclusion}] fasst die Ergebnisse der Arbeit zusammen und stellt Anknüpfungspunkte vor.
%\end{description}


%\section{State of the Art}

%\blindtext
%In the following, I will review the state of the art with respect to the topic goals and specifically focus on the currently available code-based and model-based performance problem detection approaches and their applications to locate and remove performance problems.
%Current work in the field can divided into two groups of approaches: \emph{the first group} and \emph{the second group}.


%\subsection*{The first group of approaches}
%\label{subsec_FirstGroupRelated}
%\blindtext
%\cite{AspectJ}

%\subsection*{The second group of approaches}
%\label{subsec_SecondGroupRelated}
%\blindtext
%\cite{Hoorn2009,Hoorn2012,Rohr2008}

%\section{Open Challenges Addressed by this Thesis}

%\blindtext

%In this proposal, we will develop something that nobody else does.  My approach has the advantage that it is \emph{much better than other}.  As an example, reconsider \autoref{fig:figure_circle}, but now \emph{including} something I think is important.  By quantifying this we can to that.  By capturing and quantifying doing this, we open a wide range of new applications, pushing forward the state of the art in this field.

%\begin{keypoints}
%	\item I will \underline{do something} to get that.
%	\item Based on that, I do \underline{this} and also \underline{another thing}.
%	\item I perform following: \underline{yadi}, \underline{yadi} and \underline{yada}, and achieve some \underline{goal}.
%\end{keypoints}

% !TeX spellcheck = de_DE

\chapter{Foundations and Related Work}
\label{chap:foundation}



%You need to differentiate between Foundation and related work. Foundation is a method/concept/discipline.. that you are re-using and to understand your thesis proposal, readers need to know about them. However, related work shows why this problem has not been sufficiently solved before.

\section{Foundations}
\subsection*{OAuth}
OAuth \cite{jones2012oauth} is an open standard for access delegation, commonly used to grant websites or applications limited access to user information without exposing passwords. It allows users to authorize third-party applications to access their data on another service without sharing their credentials. After the resource owner grants permission, the client receives an access token from the authorization server. This token is then used to access the protected resources on the resource server. If the access token expires, the client can use a refresh token (if provided) to obtain a new access token without requiring the resource owner to log in again.

As an OAuth consumer, Gropius can request limited access rights to external accounts (e.g., GitHub),
allowing the synchronization of issues without requiring direct password storage or management.
As an OAuth provider, Gropius would enable users to log in to other applications using their Gropius credentials,
thereby fostering greater flexibility and integration. \\
%again i am not sure if this is entirly correct*\\

%Here, introduce necessary foundations that is required for understanding your thesis. For example, your research area is on Model-driven software development, and it is necessary for readers to know the concept of model-2-model transformation.
\subsection*{Gropius}
Gropius is a state-of-the-art tool designed to address the challenges of managing issues in distributed, component-based systems. Conventional issue management systems are often constrained to a project-specific scope, which limits their effectiveness in handling cross-component dependencies. Gropius overcomes this limitation through its integrated issue management capabilities, allowing users to create and manage issues that span multiple components. This approach ensures that interdependencies between components are captured and addressed systematically, enhancing coordination and collaboration.

One of Gropius’s key features is its cross-component visibility, which documents architectural dependencies and links issues across components. This capability provides a comprehensive view of how issues propagate throughout the system, enabling better communication and understanding among development teams. Additionally, Gropius offers a user-friendly interface with a modern web-based front-end, including modules for user management, project management, and a graphical cross-component issue modeller, ensuring ease of use for both technical and nontechnical users.

Furthermore, Gropius excels in seamless integration with existing systems by acting as a wrapper over multiple issue management platforms. This design enables users to interact with Gropius rather than dealing with the complexity of each integrated system, thereby simplifying workflows. Its real-time synchronization ensures that any changes made in a component’s issue management system are instantly propagated through Gropius, keeping all stakeholders updated and aligned.

While Gropius excels in cross-component issue management, its current authentication mechanism is limited by a stateless design. Users must frequently re-authenticate, and the system lacks robust integration with external services for seamless workflows. Building on the foundational achievements of Gropius, this research focuses on enhancing its authentication infrastructure by integrating an OAuth-based system. This implementation will provide session persistence, third-party authentication (e.g., GitHub and Jira), and token-based data synchronization, addressing the identified limitations and enabling Gropius to leverage its cross-platform potential fully.
\section{Related Work}
Session management is a critical aspect of modern web applications. It ensures that user interactions are maintained across multiple requests without requiring repeated logins.
Existing solutions often rely on cookies or token-based mechanisms to manage sessions.
However, these approaches can have limitations in terms of security and scalability.
By integrating OAuth with robust session management, Gropius can enhance user experience by maintaining login states across sessions, reducing the need for frequent re-authentication,
and improving overall security through better token handling and storage mechanisms.


%In addition to discussing about foundations, you need to discuss about why the problem you are tackling has been insufficiently solved before. To do so, you need to mention related work and discuss what is the difference between your work and those related works.

% !TeX spellcheck = de_DE

\chapter{Objectives and Work Packages}
\label{chap:wp}

\section{Objectives}

The primary objective of this project is to integrate an OAuth-based authentication and synchronization service into Gropius, making it possible for users to:
\begin{itemize}
	\item Log in to Gropius using external providers (e.g., GitHub, Jira).
	\item Maintain session persistence across multiple sessions.
	\item Synchronize issue data from platforms like GitHub and Jira to Gropius using tokens with appropriate permissions.
\end{itemize}

\section{Work Packages}
\label{sec-workpackages}


\package{Analysis of Requirements and Current Architecture}

\subsubsection{Goals}

\subsubsection{Main Research Questions}


How can OAuth be effectively integrated into Gropius to provide session-based authentication and enable seamless synchronization
with external platforms like GitHub and Jira?


\subsubsection{Research Questions}
\begin{itemize}
	\item[RQ1.1]
What are the limitations of the current stateless login service in Gropius, and how can they be addressed with OAuth-based session management?
	\item[RQ1.2]
	\item[RQ1.3]
\end{itemize}

\subsubsection{Tasks}
This work package is split into the following tasks:
\begin{itemize}
	\item[T1.1] Analyze Gropius's current stateless login setup and its limitations.
	\item[T1.2] Identify requirements for OAuth integration, including user needs for external authentication and issue synchronization, as well as a Interface to the OAuth service.
	\item % TODO: Do we need to do this?
	\item[T1.3] Review OAuth flows (e.g., Authorization Code Flow, Device Flow) and their applicability to Gropius.
\end{itemize}

\package{Design of OAuth Consumer and Provider for Gropius}


\subsubsection{Goals}


\subsubsection{Research Questions}
\begin{itemize}
	\item[RQ2.1] Does evaery needed OAuth service work with the current Infrastructure of Gropius?
	\item[RQ2.2] ...
\end{itemize}

\subsubsection{Tasks}
This work package is split into the following tasks:
\begin{itemize}
	\item[T2.1] Design the OAuth consumer functionality to allow users to log in using GitHub, Google, and other OAuth-compliant providers.
	\item[T2.2] Define the session management architecture, focusing on token storage, refresh capabilities, and multi-device access.
	\item[T2.3] Plan the OAuth provider implementation for Gropius to act as an authentication provider for external applications.
\end{itemize}

\package{Implementation of OAuth Consumer and Provider Services}


\subsubsection{Goals}


\subsubsection{Research Questions}
\begin{itemize}
	\item[RQ3.1]
How can token storage and session management be securely and efficiently implemented, and
what role does the choice of database (e.g., PostgreSQL or MongoDB) play in this?
	\item[RQ3.2] ...
\end{itemize}

\subsubsection{Tasks}
This work package is split into the following tasks:
\begin{itemize}
	\item[T3.1] Implement the OAuth consumer service using NestJS and PassportJS, enabling login via third-party providers and session-based access.
	\item[T3.2] Develop the synchronization API to handle token-based issue management, supporting GitHub and, potentially, Jira synchronization.
	\item[T3.3] Ensure compatibility with PostgreSQL for efficient token and session management, and optionally evaluate MongoDB for scalability.
	\item[T3.4] (Optional) Integrate Device Flow to facilitate authentication on various devices (e.g., VSCode).
\end{itemize}

\package{Testing, Optimization, and Documentation}

Task 4.1: Conduct thorough testing of all implemented OAuth flows and synchronization mechanisms, addressing potential security vulnerabilities.
Task 4.2: Optimize the architecture based on feedback from real-world testing scenarios.
Task 4.3: Document the code, providing comprehensive internal and external documentation for maintainability and future development.

\subsubsection{Goals}


\subsubsection{Research Questions}
\begin{itemize}
	\item[RQ4.1] What are the best practices for documenting and maintaining OAuth-based authentication systems in the context of an open-source project like Gropius?
	\item[RQ4.2] ...
\end{itemize}

\subsubsection{Tasks}
This work package is split into the following tasks:
\begin{itemize}
	\item[T4.1] Conduct thorough testing of all implemented OAuth flows and synchronization mechanisms, addressing potential security vulnerabilities.
	\item[T4.2] Optimize the architecture based on feedback from real-world testing scenarios.
	\item[T4.3] Document the code, providing comprehensive internal and external documentation for maintainability and future development.
\end{itemize}

\package{Final Evaluation and Integration}

\subsubsection{Goals}


\subsubsection{Research Questions}
\begin{itemize}
	\item[RQ4.1] ...
	\item[RQ4.2] ...
\end{itemize}
\begin{itemize}
	\item[T4.1] Evaluate the OAuth integration’s performance and user experience within Gropius.
	\item[T4.2] Gather feedback from stakeholders and perform final adjustments.
	\item[T4.3] Integrate the OAuth-based service fully into Gropius and provide a summary report detailing implementation insights, challenges, and achievements.
\end{itemize}


\begin{figure}
	\begin{center}
		\begin{ganttchart}[
			x unit=0.6cm,
			y unit title=1cm,
			vgrid,hgrid]
			{1}{24}
			\gantttitle{2024}{24} \\
			\gantttitle{December}{4}
			\gantttitle{January}{4}
			\gantttitle{February}{4}
			\gantttitle{March}{4}
			\gantttitle{April}{4}
			\gantttitle{May}{4} \\

			% WP1: Analysis of Requirements and Current Architecture
			\ganttgroup{WP1}{1}{5} \\
			\ganttbar{T1.1}{1}{3} \\
			\ganttbar{T1.2}{3}{5} \\

			% WP2: Design of OAuth Consumer and Provider for Gropius
			\ganttgroup{WP2}{5}{10} \\
			\ganttbar{T2.1}{5}{7} \\
			\ganttbar{T2.2}{7}{10} \\

			% WP3: Implementation of OAuth Services
			\ganttgroup{WP3}{10}{18} \\
			\ganttbar{T3.1}{10}{14} \\
			\ganttbar{T3.2}{12}{16} \\
			\ganttbar{T3.3}{14}{18} \\

			% WP4: Testing, Optimization, and Documentation
			\ganttgroup{WP4}{16}{24} \\
			\ganttbar{T4.1}{16}{22} \\
			\ganttbar{T4.2}{22}{24} \\

			% WP5: Final Evaluation and Integration
			\ganttgroup{WP5}{20}{24} \\
			\ganttbar{T5.1}{20}{22} \\
			\ganttbar{T5.2}{22}{24}
		\end{ganttchart}
	\end{center}
	\caption{Gantt Chart for Integrating OAuth into Gropius}\label{fig:gantt_chart}
\end{figure}

%
%
%\renewcommand{\appendixtocname}{Anhang}
%\renewcommand{\appendixname}{Anhang}
%\renewcommand{\appendixpagename}{Anhang}
%\appendix
%\input{content/latex-tipps}

%\printindex

\printbibliography

\ifdeutsch
Alle URLs wurden zuletzt am 17.\,11.\,2024 geprüft.
\else
All links were last checked on November 17, 2024.
\fi

%\pagestyle{empty}
%\renewcommand*{\chapterpagestyle}{empty}
%\Versicherung
\end{document}
