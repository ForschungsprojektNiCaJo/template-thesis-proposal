% !TeX spellcheck = de_DE

\chapter{Objectives and Work Packages}
\label{chap:wp}

\section{Objectives}

The primary objective of this project is to integrate an OAuth-based authentication and synchronization service into Gropius, making it possible for users to:
\begin{itemize}
	\item Log in to Gropius using external providers (e.g., GitHub, Jira).
	\item Maintain session persistence across multiple sessions.
	\item Synchronize issue data from platforms like GitHub and Jira to Gropius using tokens with appropriate permissions.
\end{itemize}
\subsubsection{Main Research Questions}

How can OAuth be effectively integrated into Gropius to provide session-based authentication and enable seamless synchronization
with external platforms like GitHub and Jira?

\section{Work Packages}
\label{sec-workpackages}

\package{Analysis of Requirements and Current Architecture}

\subsubsection{Goals}
Analyze the current login architecure and define the requirements for integrating OAuth.
\subsubsection{Research Questions}
\begin{itemize}
	\item[RQ1.1]
	What are the limitations of the current stateless login service in Gropius, and how can they be addressed with OAuth-based session management?
\end{itemize}

\subsubsection{Tasks}
This work package is split into the following tasks:
\begin{itemize}
	\item[T1.1] Analyze Gropius's current stateless login setup and its limitations.
	\item[T1.2] Identify requirements for OAuth integration, including user needs for external authentication and issue synchronization, as well as an interface to the OAuth service.
\end{itemize}

\package{Design of OAuth Consumer and Provider for Gropius}

\subsubsection{Goals}
Design the functionalities to enable user login and session management
\subsubsection{Research Questions}
\begin{itemize}
	\item[RQ2.1] Does every needed OAuth service work with the current Infrastructure of Gropius?
\end{itemize}

\subsubsection{Tasks}
This work package is split into the following tasks:
\begin{itemize}
	\item[T2.1] Design the OAuth consumer functionality to allow users to log in using GitHub, Jira, and possibly other OAuth-compliant providers.
	\item[T2.2] Define the session management architecture, focusing on token storage, refresh capabilities, and multi-device access.
	\item[T2.3] Plan the OAuth provider implementation for Gropius to act as an authentication provider for external applications.
\end{itemize}

\package{Implementation of OAuth Consumer and Provider Services}

\subsubsection{Goals}
Implementing the previously designed methods

\subsubsection{Research Questions}
\begin{itemize}
	\item[RQ3.1]
	How can token storage and session management be securely and efficiently implemented, and
	what role does the choice of database (e.g., PostgreSQL or MongoDB) play in this?
\end{itemize}

\subsubsection{Tasks}
This work package is split into the following tasks:
\begin{itemize}
	\item[T3.1] Implement the OAuth consumer service using NestJS and PassportJS, enabling login via third-party providers and session-based access.
	\item[T3.2] Develop the synchronization API to handle token-based issue management, supporting GitHub and, potentially, Jira synchronization.
	\item[T3.3] Ensure compatibility with PostgreSQL for efficient token and session management, and optionally evaluate MongoDB for scalability.
	%\item[T3.4] (Optional) Integrate Device Flow to facilitate authentication on various devices (e.g., VSCode).
\end{itemize}

\package{Testing, Optimization, and Documentation}
\subsubsection{Goals}
Include comprehensive tests, performance optimizations, and thorough documentation of all functions to ensure robustness, efficiency, and ease of maintenance.
\subsubsection{Research Questions}
\begin{itemize}
	\item[RQ4.1] What are the best practices for documenting and maintaining OAuth-based authentication systems in the context of an open-source project like Gropius?
\end{itemize}
\begin{itemize}
	\item[T4.1] Conduct thorough testing of all implemented OAuth flows and synchronization mechanisms, addressing potential security vulnerabilities.
	\item[T4.2] Optimize the architecture based on feedback from real-world testing scenarios.
	\item[T4.3] Document the code, providing comprehensive internal and external documentation for maintainability and future development.
\end{itemize}

\package{Final Evaluation and Presentation}

\subsubsection{Goals}
Evaluate the functionality and performance of the OAuth-based authentication system, ensuring it meets project objectives. The package will conclude with a presentation showcasing the implementation and its impact on Gropius.

\begin{itemize}
	\item[T5.1] Evaluate the OAuth integration’s performance and user experience within Gropius.
	\item[T5.2] Gather feedback.
	\item[T5.3] Prepare a presentation.
\end{itemize}


\begin{figure}
	\begin{center}
		\begin{ganttchart}[
			x unit=0.6cm,
			y unit title=1cm,
			vgrid,hgrid]
		{1}{24}
			\gantttitle{2025}{24} \\
			\gantttitle{January}{4}
			\gantttitle{February}{4}
			\gantttitle{March}{4}
			\gantttitle{April}{4}
			\gantttitle{May}{4}
			\gantttitle{June}{2}
			\gantttitle{Dec.}{2}\\

			% WP1: Analysis of Requirements and Current Architecture
			\ganttgroup{WP1}{1}{5} \\
			\ganttbar{T1.1}{1}{3} \\
			\ganttbar{T1.2}{3}{5} \\

			% WP2: Design of OAuth Consumer and Provider for Gropius
			\ganttgroup{WP2}{5}{10} \\
			\ganttbar{T2.1}{5}{7} \\
			\ganttbar{T2.2}{7}{10} \\

			% WP3: Implementation of OAuth Services
			\ganttgroup{WP3}{10}{18} \\
			\ganttbar{T3.1}{10}{14} \\
			\ganttbar{T3.2}{12}{16} \\
			\ganttbar{T3.3}{14}{18} \\

			% WP4: Testing, Optimization, and Documentation
			\ganttgroup{WP4}{16}{24} \\
			\ganttbar{T4.1}{16}{22} \\
			\ganttbar{T4.2}{22}{24} \\
			\ganttbar{T4.3}{16}{24} \\

			% WP5: Final Evaluation and Integration
			\ganttgroup{WP5}{20}{24} \\
			\ganttbar{T5.1}{20}{22} \\
			\ganttbar{T5.2}{22}{24}
			\ganttbar{T5.3}{22}{24}
		\end{ganttchart}
	\end{center}
	\caption{Gantt Chart for Integrating OAuth into Gropius}\label{fig:gantt_chart}
\end{figure}
