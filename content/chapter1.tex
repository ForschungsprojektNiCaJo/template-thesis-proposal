% !TeX spellcheck = de_DE

\chapter{Introduction}
one paragraph per item of the abstract. However, provide more details and do not just copy sentences from the abstract. 
Additionally, provide  another paragraph and  Make your own \textbf{contribution(s)} explicit ("The contribution of this thesis is...").


\section*{Thesis Structure}
Here, give an overview of your thesis proposal structure.
\begin{description}
	\item[Chapter~\ref{chap:foundation} -- \nameref{chap:foundation}:] Here, we provide...
	\item[Chapter~\ref{chap:wp} -- \nameref{chap:wp}] We conclude our thesis ...
\end{description}

%\section*{Thesis Structure}
%Die Arbeit ist in folgender Weise gegliedert:
%\begin{description}
%\item[Kapitel~\ref{chap:ch2} -- \nameref{chap:ch2}:] Hier werden werden die Grundlagen dieser Arbeit beschrieben.
%\item[Kapitel~\ref{chap:conclusion} -- \nameref{chap:conclusion}] fasst die Ergebnisse der Arbeit zusammen und stellt Anknüpfungspunkte vor.
%\end{description}


%\section{State of the Art}

%\blindtext
%In the following, I will review the state of the art with respect to the topic goals and specifically focus on the currently available code-based and model-based performance problem detection approaches and their applications to locate and remove performance problems.
%Current work in the field can divided into two groups of approaches: \emph{the first group} and \emph{the second group}.


%\subsection*{The first group of approaches}
%\label{subsec_FirstGroupRelated}
%\blindtext
%\cite{AspectJ}

%\subsection*{The second group of approaches}
%\label{subsec_SecondGroupRelated}
%\blindtext
%\cite{Hoorn2009,Hoorn2012,Rohr2008}

%\section{Open Challenges Addressed by this Thesis}

%\blindtext

%In this proposal, we will develop something that nobody else does.  My approach has the advantage that it is \emph{much better than other}.  As an example, reconsider \autoref{fig:figure_circle}, but now \emph{including} something I think is important.  By quantifying this we can to that.  By capturing and quantifying doing this, we open a wide range of new applications, pushing forward the state of the art in this field.

%\begin{keypoints}
%	\item I will \underline{do something} to get that.
%	\item Based on that, I do \underline{this} and also \underline{another thing}.
%	\item I perform following: \underline{yadi}, \underline{yadi} and \underline{yada}, and achieve some \underline{goal}.
%\end{keypoints}
