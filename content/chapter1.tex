% !TeX spellcheck = de_DE

\chapter{Introduction}


Authentication and authorization are critical components of modern software engineering, particularly in web-based applications that handle sensitive user data and integrate with external services.
OAuth \cite{SINGH2022103091}, an open standard for access delegation, has become a widely adopted solution for enabling secure authentication and seamless integration between platforms.
It allows applications to act as either consumers of third-party services (for features like social logins and data synchronization) or providers of authentication services for other systems.
This research focuses on integrating OAuth into Gropius \cite{gropius}, an issue management tool designed to unify workflows across platforms such as GitHub and Jira \cite{Atlassian, GitHub}.

The current Gropius login mechanism employs a stateless authentication service. As the current implementation does not use stateful sessions, the users have to frequently re-authenticate. \cite{GropiusGitHub}
This limitation impacts the user experience and creates barriers to integrating with external platforms, as there is no mechanism for persistent sessions.
Addressing these challenges is essential to enhance Gropius’s interoperability and provide a seamless experience for users who rely on multiple tools in their workflows.

To address these issues, this research aims to design and implement a new OAuth 2.1-based authentication system for Gropius.
This system will allow Gropius to function both as an OAuth consumer—enabling third-party logins and synchronization with external platforms like GitHub—and as an OAuth provider,
allowing other applications to authenticate users using their Gropius accounts.
The solution will incorporate session-based \cite{895139, 716693} access to overcome the limitations of the existing stateless service, ensuring that users remain logged in across sessions and devices.

The research methodology involves developing a robust authentication service that combines session management with token-based synchronization for cross-service data handling.
The implementation will consider scalability, security, and system compatibility, including the selection of a suitable database for managing session states and tokens.
This approach will enable stateful user sessions, third-party authentication (e.g., via GitHub and Jira),
and API-level data synchronization for real-time updates across integrated platforms.

The contribution of this research is the development of a comprehensive OAuth-based authentication system that enhances Gropius’s ability to integrate with external platforms and improves the overall user experience.
By enabling session persistence, supporting third-party logins, and facilitating cross-platform data synchronization,
this project positions Gropius as a more scalable, secure, and user-friendly tool for unified issue management.
This advancement will strengthen Gropius’s role as a central hub for managing workflows in diverse, interconnected environments.




\section*{Proposal Structure}
\begin{description}
	\item[Chapter~\ref{chap:foundation} -- \nameref{chap:foundation}:] Here, we provide the neccesary information about OAuth and the Gropius tool.
	\item[Chapter~\ref{chap:wp} -- \nameref{chap:wp}] Here, we present the objectives and work packages of our research project.
\end{description}

%\section*{Thesis Structure}
%Die Arbeit ist in folgender Weise gegliedert:
%\begin{description}
%\item[Kapitel~\ref{chap:ch2} -- \nameref{chap:ch2}:] Hier werden werden die Grundlagen dieser Arbeit beschrieben.
%\item[Kapitel~\ref{chap:conclusion} -- \nameref{chap:conclusion}] fasst die Ergebnisse der Arbeit zusammen und stellt Anknüpfungspunkte vor.
%\end{description}


%\section{State of the Art}

%\blindtext
%In the following, I will review the state of the art with respect to the topic goals and specifically focus on the currently available code-based and model-based performance problem detection approaches and their applications to locate and remove performance problems.
%Current work in the field can divided into two groups of approaches: \emph{the first group} and \emph{the second group}.


%\subsection*{The first group of approaches}
%\label{subsec_FirstGroupRelated}
%\blindtext
%\cite{AspectJ}

%\subsection*{The second group of approaches}
%\label{subsec_SecondGroupRelated}
%\blindtext
%\cite{Hoorn2009,Hoorn2012,Rohr2008}

%\section{Open Challenges Addressed by this Thesis}

%\blindtext

%In this proposal, we will develop something that nobody else does.  My approach has the advantage that it is \emph{much better than other}.  As an example, reconsider \autoref{fig:figure_circle}, but now \emph{including} something I think is important.  By quantifying this we can to that.  By capturing and quantifying doing this, we open a wide range of new applications, pushing forward the state of the art in this field.

%\begin{keypoints}
%	\item I will \underline{do something} to get that.
%	\item Based on that, I do \underline{this} and also \underline{another thing}.
%	\item I perform following: \underline{yadi}, \underline{yadi} and \underline{yada}, and achieve some \underline{goal}.
%\end{keypoints}
